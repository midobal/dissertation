\chapter*{Preface}\chaptermark{Preface}\addcontentsline{toc}{chapter}{Preface}\mtcaddchapter
\rput[c](0.46\textwidth,6.5){\psvectorian[scale=0.5]{24}}

This is a good place to motivate the problems you are facing on your dissertation and the proposals you made to solve them.

\pagebreak

The scientific goals of this thesis are divided into two main groups:

\begin{enumerate}
	\item \textbf{Group 1}. We propose to do things.
	\item \textbf{Group 2}. We did things!
\end{enumerate}

This dissertation is structured in 7 chapters that relate as follows:

\newlength{\introlength}\settowidth{\introlength}{test \nameref{ch:example}}\setlength{\introlength}{0.6\introlength}

\begin{figure*}[!h]
	\centering
	\footnotesize
	\resizebox{\textwidth}{!}{
	\begin{tikzpicture}[node distance=0.75cm]
		%\scriptsize
		\node (intro) [chapter, text width=\the\introlength] {Ch. \ref{ch:example} \\ \nameref{ch:example}};
		\node (mt) [chapter, text width=\the\introlength, below=0.75cm of intro] {Ch. \ref{ch:example} \\ \nameref{ch:example}};
		\node (lmod) [chapter, text width=\the\introlength, below=0.75cm of mt] {Ch. \ref{ch:example} \\ \nameref{ch:example}};
		\node (imt) [chapter, text width=\the\introlength, left=0.5cm of lmod] {Ch. \ref{ch:example} \\ \nameref{ch:example}};
		\node (snor) [chapter, text width=\the\introlength, right=0.5cm of lmod] {Ch. \ref{ch:example} \\ \nameref{ch:example}};
		\node (imthd) [chapter, text width=\the\introlength, below=of lmod] {Ch. \ref{ch:example} \\ \nameref{ch:example}};
		\node (conclusions) [chapter, text width=\the\introlength, below=of imthd] {Ch. \ref{ch:example} \\ \nameref{ch:example}};
		\draw [arrow] (intro) -- (mt);
		\draw [arrow] (mt) -| (imt);
		\coordinate [above=0.5cm of lmod](lmodpoint);
		\coordinate [above=0.5cm of snor](snorpoint);
		\coordinate [left=1.5cm of snorpoint](intropoint);
		\draw [line] (intro) -| (intropoint);
		\draw [line] (mt) -| (intropoint);
		\draw [arrow] (lmodpoint) -- (lmod);
		\draw [arrow] (snorpoint) -- (snor);
		\draw [line] (lmodpoint) -- (snorpoint);
		\draw [arrow] (imt) |- (imthd);
		\draw [arrow] (lmod) -- (imthd);
		\draw [arrow] (snor) |- (imthd);
		\draw [arrow] (imthd) -- (conclusions);
	\end{tikzpicture}
	}	
\end{figure*}

The content of each chapter is:

\begin{description}
	\item[\cref{ch:example}] showcases the usage of this dissertation template, spiced with a pinch of humor.
\end{description}

These chapters are complemented by the following appendixes:

\begin{description}
	\item[\cref{ap:appendix}] showcases how an appendix looks like.
\end{description}