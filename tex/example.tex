\chapter{This Is a Fairly Long Title to Check How it Looks Like}
\label{ch:example}
\rput[r](0.585\textwidth,2.5){\psvectorian[color=cross,scale=0.4]{58}}
\begin{Resumen}
	\QuotesFont
	\noindent Roses are red, \\
	violets are blue. \\
	I've written a thesis, \\
	and so can you.
	
	\emph{(}\textbf{Roses}. \emph{Anony Mouse}.\emph{)}
\end{Resumen}
\minitoc
\clearpage

This document showcases the usage of this dissertation template, spiced with a pinch of humor.

\section{Environments}
\label{se:env}
In this section we are going to see the different environments that are available in this document.

\subsection{Figures}
\cref{fig:phd} showcases an example of the \emph{Figure} environment.

\begin{figure}[!h]
	\centering
	\includegraphics[scale=0.3]{fig/phd.png}
	\caption[A PhD student writing their dissertation]{A PhD student writing their dissertation.}
	\label{fig:phd}
\end{figure}

\subsection{Tables}
\cref{ta:results} exemplifies how to showcase the results of your awesome proposals.

\begin{table}[!ht]
	\caption[How to showcase the results of your awesome proposals.]{How to showcase the results of your awesome proposals.}
	\label{ta:results}
	\centering
	\resizebox{0.95\textwidth}{!}{\begin{minipage}{\textwidth}
			\centering
			\begin{tabular}{c c c}
				\toprule
				\multirow{2}{*}{\textbf{Proposal}} & \textbf{Metric 1} & \textbf{Metric 2} \\
				&  \textbf{[$\downarrow$]} & \textbf{[$\uparrow$]}  \\
				\midrule
				Baseline & 0.5 & 0.5 \\
				Proposal 1 & 0.0 & 1.0 \\
				Proposal 2 & 0.0 & 1.0 \\
				\bottomrule
			\end{tabular}
	\end{minipage}}
\end{table}

\subsection{Examples}
This document comes with a custom \emph{Example} environment, which should be useful for creating... examples! Have a look at \cref{ej:example} for an exemplification that exemplifies how to exemplify an example.

\begin{Ejemplo}{\myexamples{Exemplification of how to exemplify and example.}Exemplification of how to exemplify and example.}{ej:example}
	What better way to exemplify something than using its own example environment?
	\label{ej:example}
\end{Ejemplo}

\section{Acronyms}
Another useful tool is the management of acronyms. You can define your acronyms at \texttt{tex/acros.tex}. Then, you can use them with either the command \texttt{{\backslash}ac} (for written the acronym uncapitalized) or \texttt{{\backslash}Ac} (for capitalizing the first letter, which is useful for beginning of sentences). For example, \texttt{{\backslash}ac\{vip\}} produces: \ac{vip}.